\documentclass[11pt]{exam}
\newcommand{\myname}{Xue Zhongkai} %Write your name in here
\newcommand{\myUCO}{122090636} %write your UCO in here
\newcommand{\myhwtype}{Homework}
\newcommand{\myhwnum}{1} %Homework set number
\newcommand{\myclass}{MAT2040}
\newcommand{\mylecture}{}
\newcommand{\mysection}{}

% Prefix for numedquestion's
\newcommand{\questiontype}{Question}

% Use this if your "written" questions are all under one section
% For example, if the homework handout has Section 5: Written Questions
% and all questions are 5.1, 5.2, 5.3, etc. set this to 5
% Use for 0 no prefix. Redefine as needed per-question.
\newcommand{\writtensection}{0}

\usepackage{amsmath, amsfonts, amsthm, amssymb}  % Some math symbols
\usepackage{enumerate}
\usepackage{enumitem}
\usepackage{graphicx}
\usepackage{hyperref}
\usepackage[all]{xy}
\usepackage{wrapfig}
\usepackage{fancyvrb}
\usepackage[T1]{fontenc}
\usepackage{listings}

\usepackage{centernot}
\usepackage{mathtools}
\DeclarePairedDelimiter{\ceil}{\lceil}{\rceil}
\DeclarePairedDelimiter{\floor}{\lfloor}{\rfloor}
\DeclarePairedDelimiter{\card}{\vert}{\vert}


\setlength{\parindent}{0pt}
\setlength{\parskip}{5pt plus 1pt}
\pagestyle{empty}

\def\indented#1{\list{}{}\item[]}
\let\indented=\endlist

\newcounter{questionCounter}
\newcounter{partCounter}[questionCounter]

\newenvironment{namedquestion}[1][\arabic{questionCounter}]{%
    \addtocounter{questionCounter}{1}%
    \setcounter{partCounter}{0}%
    \vspace{.2in}%
        \noindent{\bf #1}%
    \vspace{0.3em} \hrule \vspace{.1in}%
}{}

\newenvironment{numedquestion}[0]{%
	\stepcounter{questionCounter}%
    \vspace{.2in}%
        \ifx\writtensection\undefined
        \noindent{\bf \questiontype \; \arabic{questionCounter}. }%
        \else
          \if\writtensection0
          \noindent{\bf \questiontype \; \arabic{questionCounter}. }%
          \else
          \noindent{\bf \questiontype \; \writtensection.\arabic{questionCounter} }%
        \fi
    \vspace{0.3em} \hrule \vspace{.1in}%
}{}

\newenvironment{alphaparts}[0]{%
  \begin{enumerate}[label=\textbf{(\alph*)}]
}{\end{enumerate}}

\newenvironment{arabicparts}[0]{%
  \begin{enumerate}[label=\textbf{\arabic{questionCounter}.\arabic*})]
}{\end{enumerate}}

\newenvironment{questionpart}[0]{%
  \item
}{}

\newcommand{\answerbox}[1]{
\begin{framed}
\vspace{#1}
\end{framed}}

\pagestyle{head}

\headrule
\header{\textbf{\myclass\ \mylecture\mysection}}%
{\textbf{\myname\ (\myUCO)}}%
{\textbf{\myhwtype\ \myhwnum}}

\begin{document}
\thispagestyle{plain}
\begin{center}
  {\Large \myclass{} \myhwtype{} \myhwnum} \\
  \myname{} (\myUCO{}) \\
  \today
\end{center}


% Question 1
\begin{numedquestion}
  \begin{arabicparts}
    \item \textbf{False.} The system is inconsistent if it has no solutions. 
    
    \item \textbf{True.} 
    
    \item \textbf{False.} One possible non-e.g.: 
    $\begin{bmatrix} 4 & 0 \\0 & 0\end{bmatrix}
    \begin{bmatrix} 0 & 0 \\0 & 5\end{bmatrix} = 
    \begin{bmatrix} 0 & 0 \\0 & 0\end{bmatrix}.$

    \item \textbf{True.} 

    \item \textbf{False.} $AB \neq BA.$
    
  \end{arabicparts}
\end{numedquestion}

% Question 2
\begin{numedquestion}
  Written in the form of \textbf{augmented matrix}, we have  
  $$ \left[\begin{array}{ccc|c}
      1 & -1 & 2 & 1 \\
      2 & 1 & 1 & 8 \\
      1 & 1 & 0 & 5\end{array} \right]
      \xrightarrow[R_3 \rightarrow R_3-R_1]{R_2 \rightarrow \frac{R_2-2R_1}{3}}
      \left[\begin{array}{ccc|c}
      1 & -1 & 2 & 1 \\
      0 & 1 & -1 & 2 \\
      0 & 2 & -2 & 4\end{array} \right]
      \xrightarrow{R_3 \rightarrow R_3-2R_2}
      \left[\begin{array}{ccc|c}
      1 & -1 & 2 & 1 \\
      0 & 1 & -1 & 2 \\
      0 & 0 & 0 & 0\end{array} \right]
  $$ and 
  $$ \left[\begin{array}{ccc|c}
      1 & -1 & 2 & 1 \\
      2 & 1 & 1 & 8 \\
      1 & 1 & 0 & 5\end{array} \right] 
      \xrightarrow[R_3 \rightarrow R_3-R_1]{R_2 \rightarrow \frac{R_2-2R_1}{3}}
      \left[\begin{array}{ccc|c}
      1 & -1 & 2 & 1 \\
      0 & 1 & -1 & 2 \\
      0 & 2 & -2 & 4\end{array} \right] 
      \xrightarrow{R_3 \rightarrow R_3-2R_2}
      \left[\begin{array}{ccc|c}
      1 & -1 & 2 & 1 \\
      0 & 1 & -1 & 2 \\
      0 & 0 & 0 & 0\end{array} \right]
      $$
      We have identical RREFs, which indicates \textbf{equivalent} linear systems.
\end{numedquestion}

% Question 3
\begin{numedquestion}
    \begin{arabicparts}
    \item We have coefficient matrix 
    $\begin{bmatrix} 4 & 5 & 3 & 3 & 4 \\ 
                    2 & 3 & 1 & 0 & 1 \\
                    3 & 4 & 2 & 1 & 1 
    \end{bmatrix}$ and the augmented matrix 
    $\left[\begin{array}{ccccc|c}
      4 & 5 & 3 & 3 & 4 & -5\\
      2 & 3 & 1 & 0 & 1 & -3\\
      3 & 4 & 2 & 1 & 1 & -1
      \end{array} \right]$.

     \item We have
     $$\left[\begin{array}{ccccc|c}
      4 & 5 & 3 & 3 & 4 & -5\\
      2 & 3 & 1 & 0 & 1 & -3\\
      3 & 4 & 2 & 1 & 1 & -1
      \end{array} \right] \Rightarrow
      \left[\begin{array}{ccccc|c}
      2 & 3 & 1 & 0 & 1 & -3\\
      4 & 5 & 3 & 3 & 4 & -5\\
      3 & 4 & 2 & 1 & 1 & -1
      \end{array} \right] \Rightarrow
      \left[\begin{array}{ccccc|c}
      2 & 3 & 1 & 0 & 1 & -3\\
      0 & 1 & -1 & -3 & -2 & -1\\
      0 & -1 & 1 & 2 & -1 & 7
      \end{array} \right] $$
      $$\Rightarrow 
      \left[\begin{array}{ccccc|c}
      2 & 3 & 1 & 0 & 1 & -3\\
      0 & 1 & -1 & -3 & -2 & -1\\
      0 & 0 & 0 & 1 & 3 & -6
      \end{array} \right] \Rightarrow
      \left[\begin{array}{ccccc|c}
      2 & 3 & 1 & 0 & 1 & -3\\
      0 & 1 & -1 & 0 & 7 & -19\\
      0 & 0 & 0 & 1 & 3 & -6
      \end{array} \right] \Rightarrow
      \left[\begin{array}{ccccc|c}
      1 & 0 & 2 & 0 & -10 & 27\\
      0 & 1 & -1 & 0 & 7 & -19\\
      0 & 0 & 0 & 1 & 3 & -6
      \end{array} \right]$$
      $$\Rightarrow   \left \{    
    \begin{aligned}
    x_1 &= 27-2x_3+10x_5 \\
    x_2 &= -19+x_3-7x_5  \\
    x_4 &= -6-3x_5
    \end{aligned}
    \right.
      $$
    \end{arabicparts}
\end{numedquestion}

% Question 4
\begin{numedquestion}
We have the matrix to be
$$    \left[\begin{array}{ccc|c}
       3 & k & 1 & 0\\
       0 & 4 & 1 & 0 \\
       k & -5 & -1 & 0 
      \end{array} \right]$$
If $k = 0$, we have
$$\left[\begin{array}{ccc|c}
       3 & 0 & 1 & 0\\
       0 & 4 & 1 & 0 \\
       0 & -5 & -1 & 0 
      \end{array} \right] \Rightarrow 
  \left[\begin{array}{ccc|c}
       1 & 0 & 0 & 0\\
       0 & 1 & 0 & 0 \\
       0 & 0 & 1 & 0 
      \end{array} \right]
  \text{(rejected)}$$
      
If $k \neq 0$, it could be reduced to the form of
$$    \left[\begin{array}{ccc|c}
       1 & \frac{1}{3}k & \frac{1}{3} & 0\\
       0 & 1 & \frac{1}{4} & 0 \\
       0 & 0 & k^2-4k+3 & 0 
      \end{array} \right]$$
In order for the RREF to have the property that $r < n$, we have 
$$k^2-4k+3 =0$$
such that it will become a $3 \times 2$ matrix of RREF, which has infinite solutions. \\
As a result, $$k_1 = 1, \quad k_2 = 3$$
\end{numedquestion}

% Question 5
\begin{numedquestion}
We have the matrix to be
$$    \left[\begin{array}{ccc|c}
       1 & 4 & -2 & 1\\
       1 & 7 & -6 & 6 \\
       0 & 3 & p & q 
      \end{array} \right]$$
By row operations, it could be reduced to the form of
$$    \left[\begin{array}{ccc|c}
       1 & 4 & -2 & 1\\
       0 & 3 & -4 & 5 \\
       0 & 0 & p+4 & q-5
      \end{array} \right]$$
\begin{arabicparts}      
    \item For the RREF to become an inconsistent one, we have 
    $$ p+4 = 0 $$
    As a result, $$ p=-4$$
    
    \item For the RREF to have a unique solution, we have
    $$ \left \{    
    \begin{aligned}
    p+4 &\neq 0 \\
    q-5 &\neq 0 \\
    \end{aligned}
    \right.
    $$
    As a result, $$ p\neq-4$$

    \item For the RREF to have infinitely many solutions, we have
    $$ \left \{    
    \begin{aligned}
    p+4 &= 0 \\
    q-5 &= 0 \\
    \end{aligned}
    \right.
    $$
    As a result, $$ p=-4, \quad q=5$$
\end{arabicparts}      
\end{numedquestion}

% Question 6
\begin{numedquestion}
We have the matrix to be
$$\begin{bmatrix} 
   1 & 0 & -1 & 1 \\ 
   0 & 1 & 2 & 1  \\
   0 & 0 & 1 & -1 \\
   1 & 2 & 4 & b  
\end{bmatrix}
$$
By row operations, it can be reduced to the form of
$$\begin{bmatrix} 
   1 & 0 & -1 & 1 \\ 
   0 & 1 & 2 & 1  \\
   0 & 0 & 1 & -1 \\
   0 & 0 & 0 & b-2 
\end{bmatrix}
$$
By \textbf{Thm}: An underdetermined homogeneous linear system has infinite solutions, we have $$b-2=0$$
As a result, $$b=2$$
\end{numedquestion}

% Question 7
\begin{numedquestion}
We have the matrix to be
$$    \left[\begin{array}{ccc|c}
       1 & 1 & 1 & 2\\
       1 & 2 & a & -1 \\
       2 & 3 & 0 & b 
      \end{array} \right]$$
By row operations, it could be reduced to the form of
$$    \left[\begin{array}{ccc|c}
       1 & 1 & 1 & 2\\
       0 & 1 & -2 & b-4 \\
       0 & 0 & a+1 & b-1
      \end{array} \right]$$
\begin{arabicparts}      
    \item For the RREF to become an inconsistent one, we have 
    $$ \left \{
    \begin{aligned}
    a+1 &= 0 \\
    b-1 &\neq 0 \\
    \end{aligned}
    \right.
    $$
    As a result, $$ a=-1, \quad b\neq1$$
    
    \item For the RREF to have a unique solution, we have
    $$a+1 \neq 0 $$
    As a result, $$ a \neq-1$$

    \item For the RREF to have infinitely many solutions, we have
        $$ \left \{
    \begin{aligned}
    a+1 &= 0 \\
    b-1 &= 0 \\
    \end{aligned}
    \right.
    $$
    As a result, $$ a=-1,\quad b=1$$
\end{arabicparts}      
\end{numedquestion}

% Question 8
\begin{numedquestion}
We have the matrix to be
$$    \left[\begin{array}{ccc|c}
       1 & 1 & 1 & 0\\
       1 & 2 & a & 0 \\
       1 & 4 & a^2 & 0\\
       1 & 2 & 1 & a-1
      \end{array} \right]$$
By row operations, it could be reduced to the form of
$$    \left[\begin{array}{ccc|c}
       1 & 1 & 1 & 0\\
       0 & 1 & 0 & a-1 \\
       0 & 0 & a^2-3a+2 & 0\\
       0 & 0 & -1+a & 1-a
      \end{array} \right]$$
For the RREF to have at least one solution: in the case $a=1$, we have more than one solutions.
In the case $a\neq1,$ we have
$$a^2-3a+2 = 0 \Rightarrow a = 2$$
As a result, $$ a=2 \; \text{or} \; 1$$
When $a = 2$,
$$    \left[\begin{array}{ccc|c}
       1 & 1 & 1 & 0\\
       0 & 1 & 0 & 1 \\
       0 & 0 & 0 & 0\\
       0 & 0 & 1 & -1
      \end{array} \right]\Rightarrow    \left[\begin{array}{ccc|c}
       1 & 0 & 0 & 0\\
       0 & 1 & 0 & 1 \\
       0 & 0 & 1 & -1
      \end{array} \right]$$
When $a = 1$, 
$$    \left[\begin{array}{ccc|c}
       1 & 1 & 1 & 0\\
       0 & 1 & 0 & 0\\
       0 & 0 & 0 & 0\\
       0 & 0 & 0 & 0
      \end{array} \right]\Rightarrow
      \left \{\begin{aligned}
        x_1 &= -x_3\\
        x_2 &= 0 \\
       \end{aligned}\right.
$$
\end{numedquestion} 

% Question 9
\begin{numedquestion}
We find $A^2$ to be  
$$A^2 = \begin{bmatrix} 
            1 & 0 & 0 \\ 
            0 & 1 & 0 \\
            0 & 0 & 1  
\end{bmatrix} = I$$
For an odd number $k$,
$$A^k = \underbrace{(AA)(AA)...(AA)}_{\frac{k-1}{2}groups}A = I^{\frac{k-1}{2}} A = IA = A$$
For a even number $k$,
$$A^k = \underbrace{(AA)(AA)...(AA)}_{\frac{k}{2}groups} = I^{\frac{k}{2}} = I$$

\end{numedquestion} 

% Question 10
\begin{numedquestion}
\begin{arabicparts}  
    \item It is symmetric, as
    $$(A^2-B^2)^T = (A^2)^T - (B^2)^T = (A^T)^2 - (B^T)^2 = A^2 - B^2
    $$
    \item It is NOT symmetric, as we have
    $$(A+B)(A-B) = A^2 + BA - AB -B^2$$
    Given $A^2 - B^2$ is symmetric,
    $$(BA - AB)^T = (BA)^T-(AB)^T = A^TB^T -B^TA^T = AB-BA \neq BA-AB$$ 
    \item It is symmetric, as 
    $$(ABA)^T = ((AB)A)^T = A^T(AB)^T = A^TB^TA^T = ABA$$
    \item It is NOT symmetric, as 
    $$ (ABAB)^T = (AB)^T (AB)^T = B^TA^TB^TA^T = BABA \neq ABAB$$
\end{arabicparts}  
\end{numedquestion} 

\begin{numedquestion}
We have
$$A=ab^T = 
\begin{bmatrix} 
    1\\ 2\\ 2 
\end{bmatrix}
\begin{bmatrix} 
    1 & \frac{1}{2} & 0 \\  
\end{bmatrix} = 
\begin{bmatrix} 
            1 & \frac{1}{2} & 0 \\ 
            2 & 1 & 0 \\
            4 & 2 & 0
\end{bmatrix}$$ 
Thus we have 
$$A^{11} = \underbrace{(ab^T)(ab^T)...(ab^T)}_{10 groups} = a(b^T a)^10b^T = 1024 \times 
\begin{bmatrix} 
            1 & \frac{1}{2} & 0 \\ 
            2 & 1 & 0 \\
            4 & 2 & 0
\end{bmatrix} = 
\begin{bmatrix} 
            1024 & 512 & 0 \\ 
            2048 & 1024 & 0 \\
            4096 & 2048 & 0
\end{bmatrix}$$ 
\end{numedquestion}

\begin{numedquestion}
We find $A^2$ to be a diagonal matrix, as
$$A^2 = AA = \begin{bmatrix} 
            9 & 0 & 0 \\ 
            0 & 9 & 0 \\
            0 & 0 & 9
\end{bmatrix}$$
Thus $$A^8 = (A^2)^4 = 
\begin{bmatrix} 
    6561 & 0 & 0 \\ 
    0 & 6561 & 0 \\
    0 & 0 & 6561
\end{bmatrix}$$
As a result, 
$$A^8 - 6400I = 
\begin{bmatrix} 
    6561 & 0 & 0 \\ 
    0 & 6561 & 0 \\
    0 & 0 & 6561
\end{bmatrix}-6400
\begin{bmatrix} 
    1 & 0 & 0 \\ 
    0 & 1 & 0 \\
    0 & 0 & 1
\end{bmatrix}=
\begin{bmatrix} 
    161 & 0 & 0 \\ 
    0 & 161 & 0 \\
    0 & 0 & 161
\end{bmatrix}
$$
\end{numedquestion}
\begin{numedquestion}
As $A^k = 0$, we have
\begin{align*}  
& \quad\; (I-A)(I+A+A^2+...+A^{k-1}) \\
& = (I^2 +IA +IA^2 +... + IA^{k-1})+(-AI-A^2-...A^k) \\ 
&= I^2 +A+A^2+...+A^{k-1}-A-A^2-...-A^k \\
&= I^2 \\
&= I
\end{align*}
\end{numedquestion}

\begin{numedquestion}
Assume a = $\begin{bmatrix} 
    a_1\\ a_2\\ a_3
\end{bmatrix}$, 
b=$
\begin{bmatrix} 
    b_1\\ b_2\\ b_3  
\end{bmatrix}$, 
we have
$$ab^T = 
\begin{bmatrix} 
    a_1b_1 & a_1b_2 & a_1b_3 \\ 
    a_2b_1 & a_2b_2 & a_2b_3 \\
    a_3b_1 & a_3b_2 & a_3b_3
\end{bmatrix} =
\begin{bmatrix} 
    65 & 91 & 78 \\ 
    70 & 98 & 84 \\
    60 & 84 & 72
\end{bmatrix}$$
Thus we have
$$a^Tb = 
\begin{bmatrix} 
    a_1 & a_2 & a_3
\end{bmatrix}
\begin{bmatrix} 
    b_1\\ b_2\\ b_3
\end{bmatrix} = a_1b_1+a_2b_2+a_3b_3 = 65+98+72 = 235$$
\end{numedquestion}

\begin{numedquestion}
\begin{arabicparts} 
    \item $$A = 
\begin{bmatrix} 
    0 & 0 \\ 
    1 & 0 \\
\end{bmatrix}$$
    \item $$A = 
\begin{bmatrix} 
    1 & 0 \\ 
    1 & 0 \\
\end{bmatrix}$$
\end{arabicparts} 
\end{numedquestion}

\begin{numedquestion}
We have 
$$A=\begin{bmatrix} 
    2 & 0 \\ 
    3 & 0 \\
\end{bmatrix},\quad
B=\begin{bmatrix} 
    2 & 0 \\ 
    4 & 0 \\
\end{bmatrix}$$
\end{numedquestion}

\begin{numedquestion}
    For $a_{i,j}$ in matrix $A_{m \times n}$, we have
    $$A^TA = O \Rightarrow O_{ii} = \sum_{j=1}^{n} a_{ji}^2 = 0 \text{ for }i=1,2...n$$
    As $a_{ji}^2 \geq 0 $, we have
    $$a_{ji} = 0  \text{ for } j = 1, 2, 3...n$$
    As a result, 
    $$A_{ij} = 0 \quad \text{ for } i = 1,2,3...n;\; j = 1,2,3...n$$
    It is equivalent to
    $$A=O$$
\end{numedquestion}

\begin{numedquestion}
    Based on $(A+B)^2 = A^2+B^2$,
    $$(A+B)^2 = A^2+B^2+AB+BA = A^2 + B^2$$
    which indicates that $$AB + BA = 0$$
    Thus $$AB(A+I) = ABA+ABI = A(BA) + (AB)I = -A^2B+AB$$
    As $A^2 =A$, we have
    $$AB(A+I) = -AB+AB = O$$
\end{numedquestion}

\begin{numedquestion}
We have 
$$A^2 =
\begin{bmatrix} 
    1 & 0 & 0 & 0 \\ 
    0 & 1 & 0 & 0 \\
    0 & 0 & 1 & 0 \\
    0 & 0 & 0 & 1
\end{bmatrix} = I$$
Also we have
$$A^3 = A^2A = IA = A$$
Thus we have 
$$A^{2n} = I^n = I, \quad A^{2n+1} = A^{2n} A = IA = A$$
\end{numedquestion}

\begin{numedquestion}
We have
$$(I-xy^T)(I+xy^T) = I -xy^T + xy^T -(xy^T)^2 = I - (xy^T)(xy^T) = I - x(y^Tx)y^T = I$$
Thus we have
$$(I-xy^T)^{-1} = (I+xy^T)$$
\end{numedquestion}


\textbf{\\ **This is the end of Homework 1.**}
\end{document}
