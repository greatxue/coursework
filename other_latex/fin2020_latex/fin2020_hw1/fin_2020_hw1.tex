\documentclass[11pt]{exam}
\newcommand{\myname}{Xue Zhongkai} %Write your name in here
\newcommand{\myUCO}{122090636} %write your UCO in here
\newcommand{\myhwtype}{Homework}
\newcommand{\myhwnum}{1} %Homework set number
\newcommand{\myclass}{FIN2020}
\newcommand{\mylecture}{}
\newcommand{\mysection}{}

% Prefix for numedquestion's
\newcommand{\questiontype}{Question}

% Use this if your "written" questions are all under one section
% For example, if the homework handout has Section 5: Written Questions
% and all questions are 5.1, 5.2, 5.3, etc. set this to 5
% Use for 0 no prefix. Redefine as needed per-question.
\newcommand{\writtensection}{0}

\usepackage{amsmath, amsfonts, amsthm, amssymb}  % Some math symbols
\usepackage{enumerate}
\usepackage{enumitem}
\usepackage{graphicx}
\usepackage{hyperref}
\usepackage[all]{xy}
\usepackage{wrapfig}
\usepackage{fancyvrb}
\usepackage[T1]{fontenc}
\usepackage{listings}

\usepackage{centernot}
\usepackage{mathtools}
\DeclarePairedDelimiter{\ceil}{\lceil}{\rceil}
\DeclarePairedDelimiter{\floor}{\lfloor}{\rfloor}
\DeclarePairedDelimiter{\card}{\vert}{\vert}


\setlength{\parindent}{0pt}
\setlength{\parskip}{5pt plus 1pt}
\pagestyle{empty}

\def\indented#1{\list{}{}\item[]}
\let\indented=\endlist

\newcounter{questionCounter}
\newcounter{partCounter}[questionCounter]

\newenvironment{namedquestion}[1][\arabic{questionCounter}]{%
    \addtocounter{questionCounter}{1}%
    \setcounter{partCounter}{0}%
    \vspace{.2in}%
        \noindent{\bf #1}%
    \vspace{0.3em} \hrule \vspace{.1in}%
}{}

\newenvironment{numedquestion}[0]{%
	\stepcounter{questionCounter}%
    \vspace{.2in}%
        \ifx\writtensection\undefined
        \noindent{\bf \questiontype \; \arabic{questionCounter}. }%
        \else
          \if\writtensection0
          \noindent{\bf \questiontype \; \arabic{questionCounter}. }%
          \else
          \noindent{\bf \questiontype \; \writtensection.\arabic{questionCounter} }%
        \fi
    \vspace{0.3em} \hrule \vspace{.1in}%
}{}

\newenvironment{alphaparts}[0]{%
  \begin{enumerate}[label=\textbf{(\alph*)}]
}{\end{enumerate}}

\newenvironment{arabicparts}[0]{%
  \begin{enumerate}[label=\textbf{\arabic{questionCounter}.\arabic*})]
}{\end{enumerate}}

\newenvironment{questionpart}[0]{%
  \item
}{}

\newcommand{\answerbox}[1]{
\begin{framed}
\vspace{#1}
\end{framed}}

\pagestyle{head}

\headrule
\header{\textbf{\myclass\ \mylecture\mysection}}%
{\textbf{\myname\ (\myUCO)}}%
{\textbf{\myhwtype\ \myhwnum}}

\begin{document}
\thispagestyle{plain}
\begin{center}
  {\Large \myclass{} \myhwtype{} \myhwnum} \\
  \myname{} (\myUCO{}) \\
  \today
\end{center}


%Here you can enter answers to homework questions


% p1
\begin{numedquestion}
  \begin{alphaparts}
    \item We have the Lagrangian multiplier for 
    $$L(c_0, c_1, c_2, \lambda) = ln(c_0) + ln(c_1) + ln(c_2) + \lambda[Y - (p_0 c_0 + p_1c_1 + p_2c_2)]$$
    For the first-order condition of choice variables, we have
    $$\frac{\partial L}{\partial c_0} = \frac{1}{c_0}- \lambda p_0, \quad
    \frac{\partial L}{\partial c_1} = \frac{1}{c_1}- \lambda p_1, \quad
    \frac{\partial L}{\partial c_2} = \frac{1}{c_2}- \lambda p_2$$
    \item By \textbf{Kuhn-Tucker Theorem}, we have
    \begin{equation}\label{i}
    \frac{\partial L}{\partial c_i} = \frac{1}{c_i}- \lambda p_i = 0, \quad i = 0,1,2
    \end{equation}\label{ii}
    and the slackness condition
    \begin{equation}
    \lambda[Y - (p_0 c_0 + p_1c_1 + p_2c_2)] = 0
    \end{equation}
    If $\lambda = 0$, by (i) we will not have a feasible real value for $c_i$, thus $\lambda \neq 0$. \\
    We also have $$p_ic_i = \frac{1}{\lambda} \quad(\lambda \neq 0)$$
    Taking it into (ii), we have $$\lambda (Y-\frac{3}{\lambda}) = 0 \quad(\lambda \neq 0)$$
    So the solution for $\lambda^*$ is $$\lambda^* = \frac{3}{Y}$$
    With a practical meaning $Y > 0$,
    $$\lambda^* > 0$$
    As a result, the constraint binds as $\lambda^* > 0$.
    \item Taking $\lambda^* = \frac{3}{Y}$ in, we have
    $$c_i^* = \frac{Y}{3p_i}$$
    Thus we have the optimal
    $$L^*=ln(c_0c_1c_2) = ln(\frac{Y^3}{27p_0p_1p_2})$$
    The larger $p_i$ is, the smaller our result; the larger $Y$ is, the larger our result.
    \item Let's define the fraction to be $F_i$. We have
    $$F_i = \frac{c_ip_i}{Y} = \frac{1}{3}, \quad i =0,1,2$$
  \end{alphaparts}
\end{numedquestion}

% p2
\begin{numedquestion}
  \begin{alphaparts}
    \item We have the Lagrangian multiplier for 
    $$L(y, k, l) = py-rk-wl + \lambda(k^al^b - y)$$
    For the first-order condition of choice variables, we have
    $$\frac{\partial L}{\partial y} = p-\lambda, \quad
    \frac{\partial L}{\partial k} = -r+\lambda l^b ak^{a-1}, \quad
    \frac{\partial L}{\partial l} = -w+\lambda k^a bl^{b-1}$$
    \item By \textbf{Kuhn-Tucker Theorem}, we have
    \begin{equation}\label{iii}
    \frac{\partial L}{\partial y} = \frac{\partial L}{\partial k} =\frac{\partial L}{\partial l} = 0
    \end{equation}\label{iv}
    and the slackness condition
    \begin{equation}
    \lambda(k^al^b - y) = 0
    \end{equation}
    
    \item The constraint binds at the optimum when
    $$ \lambda^* > 0$$
    And we have the solution set to be 
    $$k^* = (w^b a^{b-1} b^{-b} r^{1-b} p^{-1})^{1-a-b}$$
    $$l^* = (w^{1-a} a^{b-1} b^{-b} r^a p^{-1})^{1-a-b}$$
    $$y^* = [w^b a^{(a+b-1)b-a} b^{-b(a+b)} r^a p^{-a-b}]^{1-a-b}$$
     
    \item Holding others the same:
    \begin{enumerate}[label=(\roman*)]
        \item $p$ rises $\rightarrow$ \, $y^*$, $k^*$, $l^*$ all falls.
        \item $r$ rises $\rightarrow$ \, $y^*$ increases, $k^*$ increases, and $l^*$ increases.
        \item $w$ rises $\rightarrow$ \, $y^*$ increases, $k^*$ increases, and $l^*$ falls.
        \item $p$, $r$, $w$ all double $\rightarrow$ \, $y^*$, $k^*$, $l^*$ stays the same, as the sum of powers of $p$, $r$ and $w$ is $0$.
      \end{enumerate}
  \end{alphaparts}
\end{numedquestion}

\begin{numedquestion}
    \begin{alphaparts}
    \item We have the \textit{pdf} of $X$ to be 
    $$f(x;\mu, \sigma) = \frac{1}{\sigma \sqrt{2 \pi}}\exp [-\frac{(x-\mu)^2}{2\sigma^2}]dx$$
    Thus 
    $$E[u(X)] = \int_{-\infty}^{\infty} u(x) f(x) dx = \int_{-\infty}^{\infty} (-e^{v X}) \frac{1}{\sigma \sqrt{2 \pi}}\exp [-\frac{(x-\mu)^2}{2\sigma^2}]dx$$
    Recall: the \textit{mgf} of the normal distribution is
    $$M_x(t) = e^{\mu t-\frac{1}{2}t^2\sigma^2} = E[e^{tx}]$$
    Similarly, we have 
    $$E[u(X)] = -e^{-\mu v-\frac{1}{2}v^2\sigma^2}$$
    As we have $E(X) = \mu$,
    $$u(E[X]) = u(\mu) = -e^{-\mu v}$$
    They are not equal. Obviously, we have
    $$E[u(X)] > u(E[X])$$

    \item Like what has been introduced above, we have
    $$E[u(aX)] = -e^{-\mu av-\frac{1}{2}v^2 a^2 \sigma^2}\quad(\text{parameter }v > 0)$$
    The optimization problems towards $E[u(aX)]$ and $\mu a-\frac{1}{2}v a^2 \sigma^2$ are thus \textbf{equivalent}.
    \end{alphaparts}
    \textbf{***This is the end of Homework 1.***}
\end{numedquestion}

\end{document}
