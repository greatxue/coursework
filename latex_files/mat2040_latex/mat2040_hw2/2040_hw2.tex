\documentclass[11pt]{exam}
\newcommand{\myname}{Xue Zhongkai} %Write your name in here
\newcommand{\myUCO}{122090636} %write your UCO in here
\newcommand{\myhwtype}{Homework}
\newcommand{\myhwnum}{2} %Homework set number
\newcommand{\myclass}{MAT2040}
\newcommand{\mylecture}{}
\newcommand{\mysection}{}

% Prefix for numedquestion's
\newcommand{\questiontype}{Question}

% Use this if your "written" questions are all under one section
% For example, if the homework handout has Section 5: Written Questions
% and all questions are 5.1, 5.2, 5.3, etc. set this to 5
% Use for 0 no prefix. Redefine as needed per-question.
\newcommand{\writtensection}{0}
\usepackage{arydshln}
\usepackage{amsmath, amsfonts, amsthm, amssymb}  % Some math symbols
\usepackage{enumerate}
\usepackage{enumitem}
\usepackage{graphicx}
\usepackage{hyperref}
\usepackage[all]{xy}
\usepackage{wrapfig}
\usepackage{fancyvrb}
\usepackage[T1]{fontenc}
\usepackage{listings}

\usepackage{centernot}
\usepackage{mathtools}
\DeclarePairedDelimiter{\ceil}{\lceil}{\rceil}
\DeclarePairedDelimiter{\floor}{\lfloor}{\rfloor}
\DeclarePairedDelimiter{\card}{\vert}{\vert}


\setlength{\parindent}{0pt}
\setlength{\parskip}{5pt plus 1pt}
\pagestyle{empty}

\def\indented#1{\list{}{}\item[]}
\let\indented=\endlist

\newcounter{questionCounter}
\newcounter{partCounter}[questionCounter]

\newenvironment{namedquestion}[1][\arabic{questionCounter}]{%
    \addtocounter{questionCounter}{1}%
    \setcounter{partCounter}{0}%
    \vspace{.2in}%
        \noindent{\bf #1}%
    \vspace{0.3em} \hrule \vspace{.1in}%
}{}

\newenvironment{numedquestion}[0]{%
	\stepcounter{questionCounter}%
    \vspace{.2in}%
        \ifx\writtensection\undefined
        \noindent{\bf \questiontype \; \arabic{questionCounter}. }%
        \else
          \if\writtensection0
          \noindent{\bf \questiontype \; \arabic{questionCounter}. }%
          \else
          \noindent{\bf \questiontype \; \writtensection.\arabic{questionCounter} }%
        \fi
    \vspace{0.3em} \hrule \vspace{.1in}%
}{}

\newenvironment{alphaparts}[0]{%
  \begin{enumerate}[label=\textbf{(\alph*)}]
}{\end{enumerate}}

\newenvironment{arabicparts}[0]{%
  \begin{enumerate}[label=\textbf{\arabic{questionCounter}.\arabic*})]
}{\end{enumerate}}

\newenvironment{questionpart}[0]{%
  \item
}{}

\newcommand{\answerbox}[1]{
\begin{framed}
\vspace{#1}
\end{framed}}

\pagestyle{head}

\headrule
\header{\textbf{\myclass\ \mylecture\mysection}}%
{\textbf{\myname\ (\myUCO)}}%
{\textbf{\myhwtype\ \myhwnum}}

\begin{document}
\thispagestyle{plain}
\begin{center}
  {\Large \myclass{} \myhwtype{} \myhwnum} \\
  \myname{} (\myUCO{}) \\
  \today
\end{center}


% Question 1
\begin{numedquestion} 
    Define the matrix to be
        $B = 
        \begin{bmatrix} 
        a & b \\ 
        b & c 
        \end{bmatrix} (a, b, c \neq 0),$ then
        $$B^2 = 
        \begin{bmatrix} 
        a^2 + b^2 & ab + bc \\ 
        ab + bc & b^2 +c^2
        \end{bmatrix} .$$ 
    If $B^2 = O$, we have
    $$ \left \{    
    \begin{aligned}
    a^2 + b^2 &\neq 0 \\
    ab + bc &\neq 0 \\
    b^2 +c^2 &\neq 0
    \end{aligned}
    \right. \Rightarrow
    \left \{    
    \begin{aligned}
    a &\neq 0 \\
    b &\neq 0 \\
    c &\neq 0
    \end{aligned}
    \right.,
    $$
    \hspace{1em}which is a contradiction to the basic assumption $a, b, c \neq 0$.
    
    Thus it is NOT possible for a nonzero symmetric $2 \times 2$ matrix to have this property.
\end{numedquestion}

% Question 2
\begin{numedquestion}
    \begin{alphaparts}
        \item  We have 
        $$A^T = (C+C^T)^T = C^T +(C^T)^T = C^T + C = A$$
        Thus it must be necessarily symmetric.
    
        \item We have
        $$B^T = (C-C^T)^T = C^T -(C^T)^T = C^T - C \neq A$$
        Thus it could possibly be non-symmetric.
    
        \item We have
        $$D^T = (C^TC)^T = C^T (C^T)^T = C^TC = D$$
        Thus it must be necessarily symmetric.

        \item We have
        $$E^T = (C^TC-CC^T)^T = (C^TC)^T-(CC^T)^T = C^TC-CC^T = E$$
        Thus it must be necessarily symmetric.

        \item We have
        $$F^T = [(I+C)(I+C^T)]^T = (I+C^T)^T(I+C)^T = (I+C)(I+C^T) = F$$
        Thus it must be necessarily symmetric.

        \item We have
        $$F^T = [(I+C)(I-C^T)]^T = (I-C^T)^T(I+C)^T = (I-C)(I+C^T) = F$$
        Thus it could possibly be non-symmetric.

    \end{alphaparts}
\end{numedquestion}  

% Question 3
\begin{numedquestion}
We have the assumption that 
    $$M=\frac{1}{d} \cdot
    \begin{bmatrix} 
        a_{22} & -a_{12} \\ 
        -a_{21} & a_{11} 
    \end{bmatrix}$$
It is equivalent to show
    $$A^{-1} = M,$$
    \hspace{1em}which is equivalent to
    $$MA = I$$
Thus we have
    \begin{align*}
     MA &= \frac{1}{d} \cdot
    \begin{bmatrix} 
        a_{22} & -a_{12} \\ 
        -a_{21} & a_{11} 
    \end{bmatrix}
    \begin{bmatrix} 
        a_{11} & a_{12} \\ 
        a_{21} & a_{22} 
    \end{bmatrix} \\
    &= \frac{1}{a_{11}a_{22} - a_{21}a_{12}}
    \begin{bmatrix} 
        a_{22} & -a_{12} \\ 
        -a_{21} & a_{11} 
    \end{bmatrix}
    \begin{bmatrix} 
        a_{11} & a_{12} \\ 
        a_{21} & a_{22} 
    \end{bmatrix} \\
    &= \frac{1}{a_{11}a_{22} - a_{21}a_{12}}
    \begin{bmatrix} 
        a_{22}a_{11}-a_{12}a_{21} & 0 \\ 
        0 & a_{22}a_{11}-a_{12}a_{21} 
    \end{bmatrix}  \\
    &= \begin{bmatrix} 
        1 & 0 \\ 
        0 & 1 
    \end{bmatrix} \\
    &= I
    \end{align*}
    As a result, the proposition holds true.
\end{numedquestion}  

% Question 4
\begin{numedquestion}
    If $A$ is NOT singular, then there exists $A^{-1}$.
    
    Multiplying $A^{-1}$ on both sides,
    $$A^{-1}AB = A^{-1}A$$
    \hspace{1em}thus we have
    $$B=I$$
    It is a contradiction to $B \neq I$, thus A must be singular.
\end{numedquestion}

% Question 5
\begin{numedquestion}
    If $A$ is non-singular, then
    $$AB = I$$
    It is equivalent to
    $$(AB)^T = I^T$$
    \hspace{1em}then $$B^TA^T = I$$
    Thus $A^T$ is non-singular as well.
    We have 
    $$A^T(A^{-1})^T = (A^{-1}A)^T = I^T = I$$
    \hspace{1em}and
    $$(A^{-1})^TA^T = (AA^{-1})^T = I^T = I$$
    As the transpose of a matrix is unique,
    we have
    $$(A^{-1})^T = (A^T)^{-1}$$
\end{numedquestion}

% Question 6
\begin{numedquestion}
    We have
    $$(A^TA)^T = A^T (A^T)^T = A^TA$$
    \hspace{1em}and
    $$(AA^T)^T = (A^T)^TA^T = AA^T$$
    Thus $A^TA$ and $AA^T$ are both symmetric.
\end{numedquestion}

% Question 7
\begin{numedquestion}
    \begin{alphaparts}
        \item We have the elementary row operation to be
        $$R_3 \rightarrow R_1+R_3$$
        Thus we have  
    $$E=\begin{bmatrix} 
            1 & 0 & 0  \\ 
            0 & 1 & 0  \\
            1 & 0 & 1   
        \end{bmatrix}$$

        \item We have the elementary row operation to be
        $$R_2 \rightarrow -R_3+R_2$$
        Thus we have 
    $$E=\begin{bmatrix} 
            1 & 0 & 0  \\ 
            0 & 1 & -1  \\
            0 & 0 & 1   
        \end{bmatrix}$$
        \item For a given matrix $A$, performing row operations for $A$ is equivalent to pre-multiplying $A$ by the corresponding elementary matrix, thus $C$ must be row equivalent to $A$.
    \end{alphaparts}
\end{numedquestion}


% Question 8
\begin{numedquestion}
    \begin{alphaparts}
    \item We have the matrix to be
      $$\begin{bmatrix} 
            2 & 1 & 1  \\ 
            6 & 4 & 5  \\
            4 & 1 & 3   
        \end{bmatrix} \xrightarrow[R_3 \rightarrow -2R_1 + R_3]{R_2 \rightarrow -3R_1 + R_2}
        \begin{bmatrix} 
            2 & 1 & 1  \\ 
            0 & 1 & 2  \\
            0 & -1 & 1   
        \end{bmatrix} \xrightarrow{R_3 \rightarrow R_2 + R_3}
        \begin{bmatrix} 
            2 & 1 & 1  \\ 
            0 & 1 & 2  \\
            0 & 0 & 3  
        \end{bmatrix} = U$$
   The corresponding elementary matrices are
   $$E_1 = \begin{bmatrix} 
            1 & 0 & 0  \\ 
            -3 & 1 & 0  \\
            0 & 0 & 1    
        \end{bmatrix},\quad
     E_2 = \begin{bmatrix} 
            1 & 0 & 0  \\ 
            0 & 1 & 0  \\
            -2 & 0 & 1    
        \end{bmatrix},\quad
     E_3 = \begin{bmatrix} 
            1 & 0 & 0  \\ 
            0 & 1 & 0  \\
            0 & 1 & 1    
        \end{bmatrix}$$
    \item We have inverse of matrices $E_1, E_2, E_3$ to be
    $$\left \{    
    \begin{aligned}
    E_1 &= E_{-3R_1 + R_2} \\
    E_2 &= E_{-2R_1 + R_3} \\
    E_3 &= E_{R_2 + R_3}
    \end{aligned}
    \right. \Rightarrow
    \left \{    
    \begin{aligned}
    E_1^{-1} &= E_{3R_1 + R_2} \\
    E_2^{-1} &= E_{2R_1 + R_3} \\
    E_3^{-1} &= E_{-R_2 + R_3}
    \end{aligned}  
    \right. 
    $$
    The corresponding elementary matrices are
   $$E_1^{-1} = \begin{bmatrix} 
            1 & 0 & 0  \\ 
            3 & 1 & 0  \\
            0 & 0 & 1    
        \end{bmatrix},\quad
     E_2^{-1} = \begin{bmatrix} 
            1 & 0 & 0  \\ 
            0 & 1 & 0  \\
            2 & 0 & 1    
        \end{bmatrix},\quad
     E_3^{-1} = \begin{bmatrix} 
            1 & 0 & 0  \\ 
            0 & 1 & 0  \\
            0 & -1 & 1    
        \end{bmatrix}$$
    We have 
    $$L = E_1^{-1} E_2^{-1} E_3^{-1} = 
       \begin{bmatrix} 
            1 & 0 & 0 \\ 
            3 & 1 & 0  \\
            2 & -1 & 1    
        \end{bmatrix}$$
    For verification, we have
    $$L U = \begin{bmatrix} 
            1 & 0 & 0 \\ 
            3 & 1 & 0  \\
            2 & -1 & 1    
        \end{bmatrix}
        \begin{bmatrix} 
            2 & 1 & 1  \\ 
            0 & 1 & 2  \\
            0 & 0 & 3  
        \end{bmatrix}
        = \begin{bmatrix} 
            2 & 1 & 1  \\ 
            6 & 4 & 5  \\
            4 & 1 & 3  
        \end{bmatrix} = A
    $$
    \end{alphaparts}
\end{numedquestion}

% Question 9
\begin{numedquestion}
    \begin{alphaparts}
        \item First we calculate $A^{-1}$, as
        $$ [A|I] =
        \left[\begin{array}{cc|cc}
       2 & 1 & 1 & 0 \\
       6 & 4 & 0 & 1
        \end{array} \right] \Rightarrow 
        \left[\begin{array}{cc|cc}
       1 & 0 & 2 & -\frac{1}{2} \\
       0 & 1 & -3 & 1
        \end{array} \right]$$
    We have row operations as
    $$A^{-1} = 
        \begin{bmatrix} 
            2 & -\frac{1}{2} \\ 
            -3 & 1   
        \end{bmatrix} \xrightarrow{R_1 \rightarrow \frac{1}{2}R_1}
        \begin{bmatrix} 
            1 & -\frac{1}{4} \\ 
            -3 & 1   
        \end{bmatrix} \xrightarrow{R_2 \rightarrow 3R_1 + R_2}
        \begin{bmatrix} 
            1 & -\frac{1}{4} \\ 
            0 & \frac{1}{4}  
        \end{bmatrix} \xrightarrow{R_2 \rightarrow 4R_2}
        \begin{bmatrix} 
            1 & -\frac{1}{4} \\ 
            0 & 1
        \end{bmatrix} \xrightarrow{R_1 \rightarrow \frac{1}{4}R_2 + R_1}
        \begin{bmatrix} 
            1 & 0 \\ 
            0 & 1
        \end{bmatrix}$$
    Thus we have
    $$A^{-1} = E_{\frac{1}{2}R_1}^{-1} \cdot E_{3R_1 + R_2}^{-1} \cdot E_{4R_2}^{-1} \cdot E_{\frac{1}{4}R_2 + R_1}^{-1} $$
    That is, 
    $$A^{-1} = 
        \begin{bmatrix} 
            -\frac{1}{2} & 0 \\ 
            0 & 1  
        \end{bmatrix}
        \begin{bmatrix} 
            1 & 0 \\ 
            -3 & 1  
        \end{bmatrix}
        \begin{bmatrix} 
            1 & 0 \\ 
            0 & -4  
        \end{bmatrix}
        \begin{bmatrix} 
            1 & -\frac{1}{4} \\ 
            0 & 1  
        \end{bmatrix}$$
    
        \item We have
    $$A = \begin{bmatrix} 
            2 & 1 \\ 
            6 & 4  
        \end{bmatrix}
        \xrightarrow{R_1 \rightarrow \frac{1}{2}R_1}
        \begin{bmatrix} 
            1 & \frac{1}{2} \\ 
            6 & 4
        \end{bmatrix} \xrightarrow{R_2 \rightarrow -6R_1 + R_2}
        \begin{bmatrix} 
            1 & \frac{1}{2} \\ 
            0 & 1  
        \end{bmatrix} \xrightarrow{R_1 \rightarrow -\frac{1}{2}R_2 + R_1}
        \begin{bmatrix} 
            1 & -\frac{1}{4} \\ 
            0 & 1
        \end{bmatrix} \xrightarrow{R_1 \rightarrow \frac{1}{4}R_2 + R_1}
        \begin{bmatrix} 
            1 & 0 \\ 
            0 & 1
        \end{bmatrix}$$
        Thus,
        $$A = 
        \begin{bmatrix} 
            -\frac{1}{2} & 0 \\ 
            0 & 1  
        \end{bmatrix}
        \begin{bmatrix} 
            1 & 0 \\ 
            6 & 1  
        \end{bmatrix}
        \begin{bmatrix} 
            1 & 0 \\ 
            0 & \frac{1}{2}  
        \end{bmatrix}
        \begin{bmatrix} 
            1 & -\frac{1}{4} \\ 
            0 & 1  
        \end{bmatrix}$$
    \end{alphaparts}
\end{numedquestion} 

% Question 10
\begin{numedquestion}
    \begin{alphaparts}
        % a
        \item We have the row operation as 
        $$\begin{bmatrix} 
            3 & 1 \\ 
            9 & 5  
        \end{bmatrix} \xrightarrow{R_2 \rightarrow -3R_1 + R_2}
        \begin{bmatrix} 
            3 & 1 \\ 
            0 & 2  
        \end{bmatrix}$$
        Thus we have
        $$\begin{bmatrix} 
            3 & 1 \\ 
            9 & 5  
          \end{bmatrix} = 
          \begin{bmatrix} 
            1 & 0 \\ 
            3 & 1  
          \end{bmatrix} \cdot 
          \begin{bmatrix} 
            3 & 1 \\ 
            0 & 2
          \end{bmatrix}$$
        % b
        \item We have the row operation as      
        $$\begin{bmatrix} 
            2 & 4 \\ 
            -2 & 1  
        \end{bmatrix} \xrightarrow{R_2 \rightarrow R_1 + R_2}
        \begin{bmatrix} 
            2 & 4 \\ 
            0 & 5  
        \end{bmatrix}$$
        Thus we have
        $$\begin{bmatrix} 
            2 & 4 \\ 
            -2 & 1  
          \end{bmatrix} = 
          \begin{bmatrix} 
            1 & 0 \\ 
            -1 & 1  
          \end{bmatrix} \cdot 
          \begin{bmatrix} 
            2 & 4 \\ 
            0 & 5  
          \end{bmatrix}$$
        % c
        \item We have the row operations as      
        $$\begin{bmatrix} 
            1 & 1 & 1 \\ 
            3 & 5 & 6 \\
            -2 & 2 & 7
        \end{bmatrix} \xrightarrow[R_3 \rightarrow 2R_1 + R_3]{R_2 \rightarrow -3R_1 + R_2}
        \begin{bmatrix} 
            1 & 1 & 1 \\ 
            0 & 2 & 3 \\
            0 & 4 & 9  
        \end{bmatrix} \xrightarrow{R_3 \rightarrow -2R_2 + R_3}
        \begin{bmatrix} 
            1 & 1 & 1 \\ 
            0 & 2 & 3 \\
            0 & 0 & 3  
        \end{bmatrix} $$
        Thus we have
        $$\begin{bmatrix} 
            1 & 1 & 1 \\ 
            3 & 5 & 6 \\
            -2 & 2 & 7
          \end{bmatrix} = 
          \begin{bmatrix} 
            1 & 0 & 0 \\ 
            3 & 1 & 0 \\
            0 & 0 & 1
          \end{bmatrix}  
          \begin{bmatrix} 
            1 & 0 & 0 \\ 
            0 & 1 & 0 \\
            -2 & 0 & 1
          \end{bmatrix} 
          \begin{bmatrix} 
            1 & 0 & 0 \\ 
            0 & 1 & 0 \\
            0 & 2 & 1
          \end{bmatrix} \cdot
          \begin{bmatrix} 
            1 & 1 & 1 \\ 
            0 & 2 & 3 \\
            0 & 0 & 3  
          \end{bmatrix}
          $$
        % d
        \item We have the row operations as      
        $$\begin{bmatrix} 
            -2 & 1 & 2 \\ 
             4 & 1 & -2 \\
            -6 & -3 & 4
        \end{bmatrix} \xrightarrow[R_3 \rightarrow -3R_1 + R_3]{R_2 \rightarrow 2R_1 + R_2}
        \begin{bmatrix} 
            -2 & 1  & 2 \\ 
             0 & -1 & 2 \\
             0 & -6 & -2
        \end{bmatrix} \xrightarrow{R_3 \rightarrow -6R_2 + R_3}
        \begin{bmatrix} 
            -2 & 1 & 2 \\ 
            0 & -1 & 2 \\
            0 & 0 & -14  
        \end{bmatrix} $$
        Thus we have
        $$\begin{bmatrix} 
            -2 & 1 & 2 \\ 
             4 & 1 & -2 \\
            -6 & -3 & 4
          \end{bmatrix} = 
          \begin{bmatrix} 
            1 & 0 & 0 \\ 
            -2 & 1 & 0 \\
            0 & 0 & 1
          \end{bmatrix}  
          \begin{bmatrix} 
            1 & 0 & 0 \\ 
            0 & 1 & 0 \\
            3 & 0 & 1
          \end{bmatrix} 
          \begin{bmatrix} 
            1 & 0 & 0 \\ 
            0 & 1 & 0 \\
            0 & 6 & 1
          \end{bmatrix} \cdot
          \begin{bmatrix} 
            -2 & 1 & 2 \\ 
            0 & -1 & 2 \\
            0 & 0 & -14
          \end{bmatrix}
          $$
    \end{alphaparts}

% Question 11
\end{numedquestion} 
\begin{numedquestion}
    \begin{alphaparts}
        \item If $A$ is non-singular, then $$AB = I$$
        It is equivalent to
        $$(AB)^{-1} = I^{-1}$$
        \hspace{1em}then $$B^{-1}A^{-1} = I$$
        Thus $A^{-1}$ is non-singular as well. We have 
        $$A A^{-1}x
        = I$$
        \hspace{1em}and
        $$(A^{-1})^{-1}A^{-1} = (AA^{-1})^{-1} = I^{-1} = I$$
        As the inverse of a matrix is unique,
        we have
        $$(A^{-1})^{-1} = A$$
        \item If $A$ is non-singular, then
        $$AB = I$$
        It is equivalent to
        $$(AB)^T = I^T$$
        \hspace{1em}then $$B^TA^T = I$$
        Thus $A^T$ is non-singular as well.
        We have 
        $$A^T(A^{-1})^T = (A^{-1}A)^T = I^T = I$$
        \hspace{1em}and
        $$(A^{-1})^TA^T = (AA^{-1})^T = I^T = I$$
        As the transpose of a matrix is unique,
        we have
        $$(A^{-1})^T = (A^T)^{-1}$$
    \end{alphaparts}
\end{numedquestion}

% Question 12
\begin{numedquestion}
    The system has infinitely many solutions. The solution is equivalent to 
    $$A\begin{bmatrix} 
            2m \\  m \\ -4m
        \end{bmatrix} = 0$$
    \hspace{1em} and it still holds for any other $\mathbf{x}$ satisfying the given condition.
    
    If $A$ is invertible, then the system has a unique solution for $\mathbf{x}$, but we have infinite solutions.

    Thus we have infinitely many solutions, and $A$ is singular.
\end{numedquestion}

% Question 13
\begin{numedquestion}
    Since A is symmetric and non-singular, we have
    $$A^T = A$$
    \hspace{1em}and
    $$AA^{-1} = I$$
    First, we have
    $$(A^{-1})^T A^T = (AA^{-1})^T = I$$
    \hspace{1em} Substitute $A^T$ with $A$, we have
    \begin{equation}
        (A^{-1})^T A = I
    \end{equation}
    We also have 
    \begin{equation}
        A^{-1} A = I
    \end{equation}
    By (1) and (2), we have
    $$(A^{-1})^T A=A^{-1} A$$
    \hspace{1em} Multiply $A^{-1}$ on both sides, we have
    $$(A^{-1})^T =A^{-1} $$
    Thus $A^{-1}$ is also symmetric.
\end{numedquestion}

% Question 14
\begin{numedquestion}
    First prove: $B$ is row equivalent to $A$ $\Rightarrow B = MA$ ($M$ is non-singular)
    
    \hspace{1em} Since $A$ is row equivalent to $B$, there exists a set of elementary matrices such that
    $$B = E_k E_{k-1}...E_1 A$$
    \hspace{1em} As the product of elementary matrices is invertible, we can rewrite like
    $$B = MA \quad (M \text{ is non-singular})$$
    
    Also prove: $B = MA$ ($M$ is non-singular) $\Rightarrow$ $B$ is row equivalent to $A$ 
    
    \hspace{1em} As a non-singular matrix, it could be decomposed as elementary matrices like
    $$M = E_k E_{k-1}...E_1$$
    \hspace{1em} Thus we have
    $$B = E_k E_{k-1}...E_1 A$$
    \hspace{1em} which indicates that $A$ is row equivalent to $B$.
    
    Thus the proposition is proved.
\end{numedquestion}

% Question 15
\begin{numedquestion}
    \begin{alphaparts}
        \item We have the multiplication to be
        \begin{align*}
        \text{Multiplication} &=    
          \begin{bmatrix} 
            1 & 1 & 1 \\ 
            2 & 1 & 2 
          \end{bmatrix}
            \begin{bmatrix} 
            4 & -2 & 1 \\ 
            2 & 3 & 1 \\
            1 & 1 & 2
          \end{bmatrix} + 
          \begin{bmatrix} 
            -1 \\ -1 
          \end{bmatrix}
          \begin{bmatrix} 
            1 & 2 & 3  
          \end{bmatrix} \\
          &= 
          \begin{bmatrix} 
            7 & 2 & 4 \\ 
            12 & 1 & 7
          \end{bmatrix} + 
          \begin{bmatrix} 
            -1 & -2 & -3  \\
            -1 & -2 & -3  
          \end{bmatrix} \\
          &= 
          \begin{bmatrix} 
            6 & 0 & 1 \\ 
            11 & -1 & 4 
          \end{bmatrix}
          \end{align*}

          
        \item We have the multiplication to be
        $$\left[\begin{array}{cc|cc} 
            1 & 2 & 0 & -2 \\ 
            8 & 5 & 8 & -5 \\
            \hline
            3 & 2 & 3 & -2 \\
            5 & 3 & 5 & -3
          \end{array}\right]$$ 

     \item We have   
          $$\begin{bmatrix} 
                \frac{3}{5} & -\frac{4}{5} \\ 
                \frac{4}{5} & \frac{3}{5}  
            \end{bmatrix}
            \begin{bmatrix} 
                \frac{3}{5} & \frac{4}{5} \\ 
                -\frac{4}{5} & \frac{3}{5}  
            \end{bmatrix} = 
            \begin{bmatrix} 
                1 & 0 \\ 
                0 & 1  
            \end{bmatrix}
          $$
          \hspace{1em} and
          $$\begin{bmatrix} 
            1 & 0
          \end{bmatrix}
          \begin{bmatrix} 
            1 \\ 0 
          \end{bmatrix} \\= 
          1$$ 
          Thus we have the multiplication to be
          $$\left[\begin{array}{cc|cc} 
            1 & 0 & 0  \\ 
            0 & 1 & 0  \\
            \hline
            0 & 0 & 1  \\
            
          \end{array}\right]$$ 

        \item We have 
           
          $$\begin{bmatrix} 
            0 & 0 & 1 \\ 
            0 & 1 & 0  \\
            1 & 0 & 0
          \end{bmatrix}
            \begin{bmatrix} 
            1 & -1 \\ 
            2 & -2 \\
            3 & -3
          \end{bmatrix} = 
          \begin{bmatrix} 
            3 & -3 \\ 
            2 & -2 \\
            1 & -1
          \end{bmatrix}
          $$
          \hspace{1em} and
          $$
          \begin{bmatrix} 
            0 & 1 \\ 
            1 & 0 
          \end{bmatrix}
          \begin{bmatrix} 
            4 & -4 \\ 
            5 & -5
          \end{bmatrix} =
          \begin{bmatrix} 
            5 & -5   \\
            4 & -4  
          \end{bmatrix} $$
          Thus we have the multiplication to be
          $$\left[\begin{array}{cc} 
            3 & -3  \\ 
            2 & -2  \\
            1 & -1  \\
            \hline
            5 & -5  \\
            4 & -4  
          \end{array}\right]$$ 
    \end{alphaparts}
\end{numedquestion}

% Question 16
\begin{numedquestion}
    Given $B$ is singular, there exists a non-zero vector$\mathbf{x}$ such that
    $$B\mathbf{x}=0$$
    If $C=AB$, then we have
    $$C\mathbf{x} = AB\mathbf{x} = A0 = 0$$
    Thus $C$ is also singular.
\end{numedquestion}

% Question 17
\begin{numedquestion}
    \begin{alphaparts}
    \item  Since $A_{11}$ and $A_{12}$ are non-singular,
    we define $$C=-A_{11}^{-1} A_{12} A_{22}^{-1}$$
    Hence we have
    $$\begin{bmatrix} 
        A_{11}^{-1} & C \\ 
        0 & A_{22}^{-1}
      \end{bmatrix}
      \begin{bmatrix} 
        A_{11} & A_{12} \\ 
        0 & A_{22}
      \end{bmatrix}=
      \begin{bmatrix} 
        A_{11} & A_{12} \\ 
        0 & A_{22}
      \end{bmatrix}
      \begin{bmatrix} 
        A_{11}^{-1} & C \\ 
        0 & A_{22}^{-1}
      \end{bmatrix} = I_{2n}$$
      
      \item As stated, $$C=-A_{11}^{-1} A_{12} A_{22}^{-1}$$
      \end{alphaparts}
\end{numedquestion}

% Question 18
\begin{numedquestion}
    We have
    $$A = LU = 
    \begin{bmatrix} 
        1 & 0 & 0 \\ 
        -1 & 1 & 0 \\
        -2 & -3 & 1
      \end{bmatrix}  
      \begin{bmatrix} 
        1 & -1 & 1 \\ 
        0 & 2 & -2 \\
        0 & 0 & 3
      \end{bmatrix}$$
     \hspace{1em} and $$U = D\hat{U}=
        \begin{bmatrix} 
            1 & 0 & 0 \\ 
            0 & 2 & 0 \\
            0 & 0 & 3
          \end{bmatrix}  
          \begin{bmatrix} 
            1 & -1 & 1 \\ 
            0 & 1 & -1 \\
            0 & 0 & 1
          \end{bmatrix}$$
    Hence, 
    $$A = LD\hat{U}=
    \begin{bmatrix} 
        1 & 0 & 0 \\ 
        -1 & 1 & 0 \\
        -2 & -3 & 1
      \end{bmatrix}  
      \begin{bmatrix} 
        1 & 0 & 0 \\ 
        0 & 2 & 0 \\
        0 & 0 & 3
      \end{bmatrix}  
      \begin{bmatrix} 
        1 & -1 & 1 \\ 
        0 & 1 & -1 \\
        0 & 0 & 1
      \end{bmatrix}$$
    \\
    \textbf{**This is the end of Homework 2.**}
\end{numedquestion}



\end{document}
