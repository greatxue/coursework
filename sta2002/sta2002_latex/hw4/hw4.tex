\documentclass[12pt, a4paper, oneside]{article}
\usepackage{amsmath, amsthm, subcaption, float,amssymb, bm, graphicx, hyperref, mathrsfs, geometry}

\title{\large\textbf{STA2002 - Homework 4}}
\author{Xue Zhongkai 122090636}
\date{}
\linespread{1.5}
\geometry{a4paper, scale=0.9}
\newcounter{problemname}
\newenvironment{problem}{\stepcounter{problemname}\par\noindent\textsc{Problem \arabic{problemname}. }}{\\\par}

\begin{document}

\maketitle
% Problem 1
\begin{problem}
    \\
    (a) Given $X \sim N(\mu_1, \frac{\sigma^2}{n})$, the type-II error is
    $$
    \beta = P(\bar{x} > \mu_0 - \mathbf{z}_{\alpha} \frac{\sigma}{\sqrt{n}}\,|\,\mu =\mu_1) 
          = P(\bar{x}-\mu_1+\mu_1-\mu_0 > -\mathbf{z}_{\alpha} \frac{\sigma}{\sqrt{n}})
          = P(\frac{\bar{x}-\mu_1}{\sigma / \sqrt{n}} > -\mathbf{z}_{\alpha}-\frac{\mu_1-\mu_0}{\sigma / \sqrt{n}})
    $$
    By definition of the normal cdf, it is equivalent to  
    $$
    \beta = 1- \Phi(-\mathbf{z}_{\alpha}-\frac{\mu_1-\mu_0}{\sigma / \sqrt{n}})
          = \Phi(\mathbf{z}_{\alpha}+\frac{\mu_1-\mu_0}{\sigma / \sqrt{n}})
    $$
    With the symmetry of the bell-like cdf,
    $$ \mathbf{z}_{\beta} = 2 \cdot \frac{\mu_1-\mu_0}{\sigma / \sqrt{n}} - (\mathbf{z}_{\alpha}+\frac{\mu_1-\mu_0}{\sigma / \sqrt{n}})
                          = -\mathbf{z}_{\alpha}+\frac{\mu_1-\mu_0}{\sigma / \sqrt{n}}$$
    That is equivalent to,
    $$ n = (\mathbf{z}_{\alpha}+\mathbf{z}_{\beta})^2 \frac{\sigma^2}{(\mu_1-\mu_0)^2}$$                     
    \\
    (b)With the requirements, while $\mathbf{z}_{0.05}= -1.6449, \; \mathbf{z}_{0.025}=-1.9600$
    $$ n \geq (-1.6449-1.9600)^2 \times \frac{0.12^2}{(-0.02)^2} = 467.86 \approx 468 $$
    \textbf{The least value for n should be 468.}
\end{problem}

% Problem 2
\begin{problem}
    \\
    Set the test statistic to be 
    $$\mathbf{z} = \frac{cX + dY - \sigma}{\sqrt{\frac{\sigma^2}{m}+\frac{\sigma^2}{n}}} \sim N(0, 1)$$
    For two-sided test, the reject region is
    $$|\mathbf{z}| > \mathbf{z}_{\alpha /2}$$
\end{problem}

% Problem 3
\begin{problem}
    \\
    (a) Since $X_i$ are independent to each other, the joint pdf of $i$th and $j$th order statistics is
    $$f_{X_{(i)}, X_{(j)}}(x, y) = \frac{n![F(x)]^{i-1}}{(i-1)!(j-i-1)!(n-j)!} \cdot [F(y)-F(x)]^{j-i-1} \cdot [1-F(y)]^{n-j} \cdot f(x) \cdot f(y)$$
    Then joint pdf of $(X_{(1)},X_{(n)})$ is 
    $$f_{X_{(1)}, X_{(n)}}(u_1, u_2) = \frac{n!}{(n-2)!} \cdot [F(u_2)-F(u_1)]^{n-2}  \cdot f(u_1) \cdot f(u_2), \; u_1<u_2$$
    Given the sample range stastistics $R = x_{(n)}-x_{(1)}$, 
    $$  F(r) = P(R \leq r) = P(x_{(n)}-x_{(1)} \leq r) 
             = \iint_{u_2-u_1 \leq r}f_{X_{(1)}, X_{(n)}}(u_1, u_2)\,du_1du_2
    $$
    To compute this double integral,
    $$  F(r) = \int_{-\infty}^{\infty} \int_{u_1}^{u_1+r}\frac{n!}{(n-2)!} \cdot [F(u_2)-F(u_1)]^{n-2}  \cdot p(u_1) \cdot p(u_2)\, du_2du_1 $$
    Set $y = F(u_2)-F(u_1)$, then $\partial y = f(u_2)\partial u_2$. Plug it in,
    $$ \int_{u_1}^{u_1+r} y^{n-2}  \cdot p(u_1) \,dy = \frac{1}{n-1} [F(u_1 +r)-F(u_1)]^{n-1}p(u_1)$$
    That is why
    $$  F(r) = \int_{-\infty}^{\infty} \frac{n!}{(n-2)!} \cdot \frac{1}{n-1} \cdot [F(u_1 +r)-F(u_1)]^{n-1}p(u_1)\,du_1 
             = n \int_{-\infty}^{\infty}  [F(u_1 +r)-F(u_1)]^{n-1}p(u_1)\,du_1 ,\; r>0 $$
    which is exactly the same as given.    
    \\
    (b) Set $v = x_{(1)}$, then $x_{(n)}=R+v$, we have pdf 
    $$ f_{R,v}(r,v) = n(n-1) \cdot [F(r+v)-F(v)]^{n-2}$$  
    Given the fact it follows $U(0,1)$,
    $$ f_{R,v}(r,v) = n(n-1) \cdot [r+v-v]^{n-2} = n(n-1) \cdot r^{n-2}$$  
    As $r+v<1$, further we have
    $$ f_{R}(r) = \int_{0}^{1-r} n(n-1) \cdot r^{n-2},\,dv = n(n-1) \cdot r^{n-2},\; 0<r<1$$  
    To make it clear,
    $$ f_{R}(r) = \frac{r^{(n-1)-1} \cdot (1-r)^{2-1}}{\frac{(n-2)! \cdot 1!}{n!}}= \frac{r^{(n-1)-1}\cdot(1-r)^{2-1}}{\frac{\Gamma(n-1) \cdot \Gamma(2)}{\Gamma(n+1)}}
                = \frac{r^{(n-1)-1}\cdot(1-r)^{2-1}}{\beta(n-1,2)}$$
    Thus, it is true that $$R \sim \mathbf{Beta}(n-1,2)$$
    The expectation for the Beta distribution is
    $$ \mathbf{E}(R) = \frac{\alpha}{\alpha+\beta} = \frac{n-1}{n+1}$$
\end{problem}

% Problem 4
\begin{problem}
    \\
    The mean for the word "also" of the assumed Poisson distribution in author's 200 passages is
    $$ \lambda = \frac{0\times22 + 1\times53 + 2\times58 + 3\times39 + 4\times20 + 5\times5 + 6\times2 + 7\times1}{200} 
               = \frac{41}{20} = 2.05 
    $$
    From the Poisson distribution with parametre 2.05, we may compute $p_i$, the hypothesized probability associated with $i$th interval:
    \begin{align*}
    &p_1 = P(X=0) =\frac{e^{-2.05}(2.05)^0}{0!} = 0.1287 \\
    &p_2 = P(X=1) =\frac{e^{-2.05}(2.05)^1}{1!} = 0.2639 \\
    &p_3 = P(X=2) =\frac{e^{-2.05}(2.05)^2}{2!} = 0.2705 \\
    &p_4 = P(X=3) =\frac{e^{-2.05}(2.05)^3}{3!} = 0.1848 \\
    &p_5 = P(X=4) =\frac{e^{-2.05}(2.05)^4}{4!} = 0.0947 \\
    &p_6 = P(X\geq5) = 1 - \sum_{i=1}^{6} p_i  =  0.0574 
    \end{align*}
    The last two are combined for too small $E_i$. By calculating $E_i = np_i$, the expected frequencies follow:
    \begin{table}[!h]
        \centering
        \begin{tabular}{|l|l|l|}
        \hline
            Number of "also"  & Observed frequeny & Expected frequency \\ \hline
            0 & 22 & 25.74  \\ \hline
            1 & 53 & 52.78  \\ \hline
            2 & 58 & 54.10  \\ \hline
            3 & 39 & 36.96  \\ \hline
            4 & 20 & 18.94  \\ \hline
       $\geq$5 & 8  & 11.48  \\ \hline
        \end{tabular}
    \end{table}
    \\
    Set the null hypothesis $H_0$: The distribution is Poisson, 
    with $H_1$: The distribution is NOT Poisson.\\
    Assume the test stastic to be 
        $$\chi^2 = \sum_{i=0}^{5} \frac{(O_i-E_i)^2}{E_i} $$
    With stastics listed above, 
    $$\chi^2 = \frac{(22-25.74)^2}{25.74}+\frac{(53-52.78)^2}{52.78}+\frac{(58-54.10)^2}{54.10}\\
                    +\frac{(39-36.96)^2}{36.96}+\frac{(20-18.94)^2}{18.94}+\frac{(8-11.48)^2}{11.48}=2.0523\\
    $$
    With rejection region $$ \chi^2 < \chi^2_{0.05}(6-1) = 11.070,$$
    we fail to reject $H_0$.\\
    \textbf{As a result, it is reasonable to assume it to be a Poisson distribution.}
\end{problem}


% Problem 5
\begin{problem}
    \\
    (a) Use 30 scores as a sample, we have the sample mean and the sample variable 
    $$ \hat{\mu} = \bar{X} = \frac{1}{n}\sum_{i=1}^{n} x_i = 64.9$$
    $$ \hat{\sigma^2} = \mathbf{Var}(X) = \frac{1}{n}\sum_{i=1}^{n}(x_i-\bar{x}) = 230.56 $$
    Set the null hypothesis $H_0$: The score is adequately described by a normal distribution.\\
    For $X \sim N(64.9, 230.56)$ with $k = 5$ cells, \\
    \begin{table}[!h]
        \centering
        \begin{tabular}{|l|l|l|}
        \hline
            Class interval  & Observed frequeny & Expected frequency \\ \hline
            $-\infty  < x < 52.12$ & 8 & 6  \\ \hline
            $52.12 \leq x < 61.05$ & 6 & 6  \\ \hline
            $61.05 \leq x < 68.75$ & 4 & 6  \\ \hline
            $68.75 \leq x < 77.70$ & 4 & 6  \\ \hline
            $77.70 \leq x <\infty$ & 8 & 6  \\ \hline
        \end{tabular}
    \end{table}  
    \\
    Check the test stastic to be 
    $$\chi^2 = \sum_{i=0}^{5} \frac{(O_i-E_i)^2}{E_i} = 2.667$$  
    Since two parametres have been estimated,
    $$\chi^2_{0.05}(5-2-1)=\chi^2_{0.05}(2) = 5.991$$
    With the fact that $$\chi^2<\chi^2_{0.05}(2),$$
    we fail to reject $H_0$.\\
    \textbf{As a result, we can reasonably assume it is a normal distribution.}
    \\ \\ 
    (b) Set the null hypothesis $H_0$: Class I and Class II are independent, $i.e.$ 
    $p_{i,j} = p_{i,\cdot} \times p_{\cdot,j}$\\
    Under $H_0$,estimate the probabilities like
    $$\hat{p_{1\cdot}} = \frac{8+5+2}{30} = 0.5 \qquad \hat{p_{2\cdot}} = \frac{4+7+4}{30} = 0.5$$
    and $$\hat{p_{\cdot1}} = \frac{8+4}{30} = 0.4 \qquad \hat{p_{\cdot2}} = \frac{5+7}{30} = 0.4 \qquad 
          \hat{p_{\cdot3}} = \frac{2+4}{30} = 0.2$$
    we have the test statistic 
    $$ \sum_{i=1}^{2} \sum_{j=1}^{3} \frac{(y_{i,j} - n \hat{p_{i,\cdot}} \hat{p_{\cdot,j}})^2}
                                          {n \hat{p_{i,\cdot}} \hat{p_{\cdot,j}}}
                                    = 2.33$$
    Under significance level $\alpha = 0.05$,
    $$ \chi^2_{0.05}(2\times1)=\chi^2_{0.05}(2)  = 5.991$$
    With the fact that $$\chi^2 < \chi^2_{0.05}(2),$$
    fail to reject $H_0$. That is, they are independent.\\
    \textbf{As a result, we can reasonably assume they are independent.}\\
\end{problem}

% Problem 6
\begin{problem}
    \\
    Set $H_0$: Working life of these four batches have no significant differences, $i.e.$ $\mu_1=\mu_2=\mu_3=\mu_4=\mu.$\\
    By analysing the illustrative data, we find it is a single-factor experiment with 4 treatments of the factor (A, B, C, D),
    each has 7, 5, 8, 6 observations respectively. That is,
    $$m=4 \qquad  n = n_1+n_2+n_3+n_4 = 7+5+8+6 = 26$$
    With a simple numerical process, we have
    \begin{table}[!h]
        \centering
        \begin{tabular}{|l|r|r|}
        \hline
            Treatments  & $\sum_{j=1}^{7} x_{ij}$ & $\sum_{j=1}^{7} x_{ij}^2$ \\ \hline
            Treatment A & 11,760 & 19,785,400  \\ \hline
            Treatment B &  8,310 & 13,828,100  \\ \hline
            Treatment C & 13,290 & 22,191,700  \\ \hline
            Treatment D &  9,410 & 14,778,700  \\ \hline
        \textbf{Totals} & 42,770 & 70,583,900  \\ \hline
        \end{tabular}
    \end{table}  
    \\
    Further we have
    \begin{align*}
    SS(TO) &= 70,583,900 - \frac{42770^2}{26} = 227,250 \\
    SS(T) &= \frac{1}{7}\times11760^2 + \frac{1}{5}\times8310^2 + \frac{1}{8}\times13290^2 + \frac{1}{6}\times9410^2 
              - \frac{42770^2}{26} = 47,399.167 \\
    SS(E) &= 227,250 - 47,399.167 = 179,850.833 
    \end{align*}
    Given the significance level $\alpha = 0.05,$
    $$F = \frac{SS(T)/(m-1)}{SS(E)/(n-m)} = \frac{47,399.167/(4-1)}{179,850.833 /(26-4)} = 1.93 < 3.0491 = F_{0.05}(3, 22)$$ 
    Thus we fail to reject $H_0$.\\
    \textbf{As a result, we can reasonably assume these four batches have NO significant differences.}\\
\end{problem}
******************************* \textbf{End of Homework 4} *******************************
\end{document}